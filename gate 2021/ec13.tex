
\let\negmedspace\undefined
\let\negthickspace\undefined
\documentclass[journal,12pt,onecolumn]{IEEEtran}
\usepackage{cite}
\usepackage{amsmath,amssymb,amsfonts,amsthm}

\usepackage{graphicx}
\usepackage{textcomp}
\usepackage{xcolor}
\usepackage{txfonts}
\usepackage{listings}
\usepackage{enumitem}
\usepackage{mathtools}
\usepackage{gensymb}
\usepackage[breaklinks=true]{hyperref}
\usepackage{tkz-euclide} % loads  TikZ and tkz-base
\usepackage{listings}
\usepackage{gvv}
\usepackage{booktabs}

%
%\usepackage{setspace}
%\usepackage{gensymb}
%\doublespacing
%\singlespacing

%\usepackage{graphicx}
%\usepackage{amssymb}
%\usepackage{relsize}
%\usepackage[cmex10]{amsmath}
%\usepackage{amsthm}
%\interdisplaylinepenalty=2500
%\savesymbol{iint}
%\usepackage{txfonts}
%\restoresymbol{TXF}{iint}
%\usepackage{wasysym}
%\usepackage{amsthm}
%\usepackage{iithtlc}
%\usepackage{mathrsfs}
%\usepackage{txfonts}
%\usepackage{stfloats}
%\usepackage{bm}
%\usepackage{cite}
%\usepackage{cases}
%\usepackage{subfig}
%\usepackage{xtab}
%\usepackage{longtable}
%\usepackage{multirow}

%\usepackage{algpseudocode}
%\usepackage{enumitem}
%\usepackage{mathtools}
%\usepackage{tikz}
%\usepackage{circuitikz}
%\usepackage{verbatim}
%\usepackage{tfrupee}
%\usepackage{stmaryrd}
%\usetkzobj{all}
%    \usepackage{color}                                            %%
%    \usepackage{array}                                            %%
%    \usepackage{longtable}                                        %%
%    \usepackage{calc}                                             %%
%    \usepackage{multirow}                                         %%
%    \usepackage{hhline}                                           %%
%    \usepackage{ifthen}                                           %%
  %optionally (for landscape tables embedded in another document): %%
%    \usepackage{lscape}     
%\usepackage{multicol}
%\usepackage{chngcntr}
%\usepackage{enumerate}

%\usepackage{wasysym}
%\documentclass[conference]{IEEEtran}
%\IEEEoverridecommandlockouts
% The preceding line is only needed to identify funding in the first footnote. If that is unneeded, please comment it out.

\newtheorem{theorem}{Theorem}[section]
\newtheorem{problem}{Problem}
\newtheorem{proposition}{Proposition}[section]
\newtheorem{lemma}{Lemma}[section]
\newtheorem{corollary}[theorem]{Corollary}
\newtheorem{example}{Example}[section]
\newtheorem{definition}[problem]{Definition}
%\newtheorem{thm}{Theorem}[section] 
%\newtheorem{defn}[thm]{Definition}
%\newtheorem{algorithm}{Algorithm}[section]
%\newtheorem{cor}{Corollary}
\newcommand{\BEQA}{\begin{eqnarray}}
\newcommand{\EEQA}{\end{eqnarray}}
\newcommand{\define}{\stackrel{\triangle}{=}}
\theoremstyle{remark}
\newtheorem{rem}{Remark}

%\bibliographystyle{ieeetr}
\begin{document}
%

\bibliographystyle{IEEEtran}


\vspace{3cm}

\title{
%	\logo{
Gate 2021 Assignment 

\large{EE:1205 Signals and Systems}

Indian Institute of Technology, Hyderabad
%	}
}
\author{Abhey Garg

EE23BTECH11202
}	


% make the title area
\maketitle



%\tableofcontents


\renewcommand{\thefigure}{\theenumi}
\renewcommand{\thetable}{\theenumi}
%\renewcommand{\theequation}{\theenumi}

\section{Question EC 13}
Two continuous random variables X and Y are related as $Y = 2X+3$ . Let $\sigma_x^2 $ and $\sigma_y^2 $ denote the variances of X and Y , respectively. The variances are related as :
\begin{enumerate}
\item[(A)] $\sigma_y^2 $ = 2$\sigma_x^2$
\item[(B)] $\sigma_y^2 $ = 4$\sigma_x^2$
\item[(C)] $\sigma_y^2 $ = 5$\sigma_x^2$
\item[(D)] $\sigma_y^2 $ = 25$\sigma_x^2$
\end{enumerate}
\section{Solution}
\begin{align}
Y = 2X+3
\end{align}
Take the laplace transform:
\begin{align}
\mathcal{L}\{Y\} = \mathcal{L}\{2X + 3\} = 2\mathcal{L}\{X\} + 3\mathcal{L}\{1\}\\
\mathcal{L}\{Y\} = 2\mathcal{L}\{X\} + 3 \frac{1}{s}
\end{align}

Now, the variance of a random variable $A$ is given by:
\begin{align}
\text{Var}(A) = \mathcal{L}\{E[A^2]\} - [\mathcal{L}\{E[A]\}]^2
\end{align}
Let $A = X$ and $B = Y$. We want to find $\text{Var}(Y)$.
\begin{align}
\text{Var}(Y) = \mathcal{L}\{E[Y^2]\} - [\mathcal{L}\{E[Y]\}]^2
\end{align}
Since $Y = 2X + 3$, we can express $E[Y]$ and $E[Y^2]$ in terms of $X$.
\begin{align}
E[Y] = 2E[X] + 3 \\
E[Y^2] = 4E[X^2] + 12E[X] + 9 
\end{align}
Substitute these into the variance formula:
\begin{align}
\text{Var}(Y) = \mathcal{L}\{4E[X^2] + 12E[X] + 9\} - [\mathcal{L}\{2E[X] + 3\}]^2
\end{align}
Let $\mathcal{L}\{E[X]\} = \mu_x$ (the mean of $X$) and $\mathcal{L}\{E[X^2]\} = \mu_{x^2}$ (the mean of $X^2$).
\begin{align}
\text{Var}(Y) = 4\mu_{x^2} + 12\mu_x + 9 - [2\mu_x + 3]^2 \\
\text{Var}(Y) = 4\mu_{x^2} - 4\mu_x^2 
\end{align}
Now, remember that $\text{Var}(X) = \mu_{x^2} - \mu_x^2$, so we can substitute this in:
\begin{align}
\text{Var}(Y) = 4(\text{Var}(X)) 
\end{align}
Finally, substitute back :
\begin{align}
\sigma_y^2  = 4\sigma_x^2
\end{align}



\end{document}