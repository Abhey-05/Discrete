
\let\negmedspace\undefined
\let\negthickspace\undefined
\documentclass[journal,12pt,twocolumn]{IEEEtran}
\usepackage{cite}
\usepackage{amsmath,amssymb,amsfonts,amsthm}

\usepackage{graphicx}
\usepackage{textcomp}
\usepackage{xcolor}
\usepackage{txfonts}
\usepackage{listings}
\usepackage{enumitem}
\usepackage{mathtools}
\usepackage{gensymb}
\usepackage[breaklinks=true]{hyperref}
\usepackage{tkz-euclide} % loads  TikZ and tkz-base
\usepackage{listings}
\usepackage{gvv}
\usepackage{booktabs}

%
%\usepackage{setspace}
%\usepackage{gensymb}
%\doublespacing
%\singlespacing

%\usepackage{graphicx}
%\usepackage{amssymb}
%\usepackage{relsize}
%\usepackage[cmex10]{amsmath}
%\usepackage{amsthm}
%\interdisplaylinepenalty=2500
%\savesymbol{iint}
%\usepackage{txfonts}
%\restoresymbol{TXF}{iint}
%\usepackage{wasysym}
%\usepackage{amsthm}
%\usepackage{iithtlc}
%\usepackage{mathrsfs}
%\usepackage{txfonts}
%\usepackage{stfloats}
%\usepackage{bm}
%\usepackage{cite}
%\usepackage{cases}
%\usepackage{subfig}
%\usepackage{xtab}
%\usepackage{longtable}
%\usepackage{multirow}

%\usepackage{algpseudocode}
%\usepackage{enumitem}
%\usepackage{mathtools}
%\usepackage{tikz}
%\usepackage{circuitikz}
%\usepackage{verbatim}
%\usepackage{tfrupee}
%\usepackage{stmaryrd}
%\usetkzobj{all}
%    \usepackage{color}                                            %%
%    \usepackage{array}                                            %%
%    \usepackage{longtable}                                        %%
%    \usepackage{calc}                                             %%
%    \usepackage{multirow}                                         %%
%    \usepackage{hhline}                                           %%
%    \usepackage{ifthen}                                           %%
  %optionally (for landscape tables embedded in another document): %%
%    \usepackage{lscape}     
%\usepackage{multicol}
%\usepackage{chngcntr}
%\usepackage{enumerate}

%\usepackage{wasysym}
%\documentclass[conference]{IEEEtran}
%\IEEEoverridecommandlockouts
% The preceding line is only needed to identify funding in the first footnote. If that is unneeded, please comment it out.

\newtheorem{theorem}{Theorem}[section]
\newtheorem{problem}{Problem}
\newtheorem{proposition}{Proposition}[section]
\newtheorem{lemma}{Lemma}[section]
\newtheorem{corollary}[theorem]{Corollary}
\newtheorem{example}{Example}[section]
\newtheorem{definition}[problem]{Definition}
%\newtheorem{thm}{Theorem}[section] 
%\newtheorem{defn}[thm]{Definition}
%\newtheorem{algorithm}{Algorithm}[section]
%\newtheorem{cor}{Corollary}
\newcommand{\BEQA}{\begin{eqnarray}}
\newcommand{\EEQA}{\end{eqnarray}}
\newcommand{\define}{\stackrel{\triangle}{=}}
\theoremstyle{remark}
\newtheorem{rem}{Remark}

%\bibliographystyle{ieeetr}
\begin{document}
%

\bibliographystyle{IEEEtran}


\vspace{3cm}

\title{
%	\logo{
Discrete Assignment 

\large{EE:1205 Signals and Systems}

Indian Institute of Technology, Hyderabad
%	}
}
\author{Abhey Garg

EE23BTECH11202
}	


% make the title area
\maketitle

\newpage

%\tableofcontents

\bigskip

\renewcommand{\thefigure}{\arabic{figure}}
\renewcommand{\thetable}{\arabic{table}}
\renewcommand{\theequation}{\arabic{equation}}

\section{Question GATE AG 14}
$y=e^{mx}+e^{-mx}$ is the solution of which differential equation?
\section{Solution}

\begin{equation}
y = e^{mx} + e^{-mx} \label{eq:eq1}
\end{equation}
The Laplace transform of $e^{mx(t)}$ is given by:

\begin{align}
\mathcal{L}\{e^{mx(t)}\} = \frac{1}{s - m}
\end{align}  

Similarly, the Laplace transform of $e^{-mx(t)}$ is:

\begin{align}
\mathcal{L}\{e^{-mx(t)}\} = \frac{1}{s + m}
\end{align}
  
Now, applying the linearity property, the Laplace transform of $y(t)$ is the sum of the Laplace transforms of the individual terms:

\begin{align}
\mathcal{L}\{y(t)\} = \mathcal{L}\{e^{mx(t)}\} + \mathcal{L}\{e^{-mx(t)}\} 
\end{align} 
\begin{align}
= \frac{1}{s - m} + \frac{1}{s + m} 
\end{align}
\begin{align}
= \frac{2s}{s^2 - m^2} \quad -m < Re(s) < m
\end{align}

Taking the inverse laplace transform : 

\begin{align}
\frac{d^2y}{dx^2} - m^2y(t) = \mathcal{L}^{-1}\left(\frac{2s}{s^2 - m^2}\right)
\end{align}

To find the inverse Laplace transform of \(\frac{2s}{s^2 - m^2}\), let's express it in partial fraction form:

\begin{equation}
\frac{2s}{s^2 - m^2} = \frac{A}{s - m} + \frac{B}{s + m}
\end{equation}

Multiplying through by the common denominator:

\begin{equation}
2s = A(s + m) + B(s - m)
\end{equation}

Now, solving for A and B:

\begin{equation}
2s = As + Am + Bs - Bm
\end{equation}

Equating coefficients:

\begin{equation}
A + B = 2
\end{equation}
\begin{equation}
A - B = 0
\end{equation}

Solving this system of equations gives \(A = B = 1\). Now, we can express the original fraction in partial fraction form:

\begin{equation}
\frac{2s}{s^2 - m^2} = \frac{1}{s - m} + \frac{1}{s + m}
\end{equation}

Now, the inverse Laplace transform of each term:

\begin{equation}
\mathcal{L}^{-1}\left\{\frac{2s}{s^2 - m^2}\right\} = \mathcal{L}^{-1}\left\{\frac{1}{s - m}\right\} + \mathcal{L}^{-1}\left\{\frac{1}{s + m}\right\}
\end{equation}

This yields the following differential equation:

\begin{equation}
\frac{d^{2}y}{dx^{2}} - m^2y(t) = \delta(t - m) + \delta(t + m)
\end{equation}

where \(\delta(t - m)\) and \(\delta(t + m)\) are Dirac delta functions.

\begin{equation}
    \frac{d^{2}y}{dx^{2}} - m^2y = 0
\end{equation}

\end{document}